\PassOptionsToPackage{unicode=true}{hyperref} % options for packages loaded elsewhere
\PassOptionsToPackage{hyphens}{url}
%
\documentclass[]{article}
\usepackage{lmodern}
\usepackage{amssymb,amsmath}
\usepackage{ifxetex,ifluatex}
\usepackage{fixltx2e} % provides \textsubscript
\ifnum 0\ifxetex 1\fi\ifluatex 1\fi=0 % if pdftex
  \usepackage[T1]{fontenc}
  \usepackage[utf8]{inputenc}
  \usepackage{textcomp} % provides euro and other symbols
\else % if luatex or xelatex
  \usepackage{unicode-math}
  \defaultfontfeatures{Ligatures=TeX,Scale=MatchLowercase}
\fi
% use upquote if available, for straight quotes in verbatim environments
\IfFileExists{upquote.sty}{\usepackage{upquote}}{}
% use microtype if available
\IfFileExists{microtype.sty}{%
\usepackage[]{microtype}
\UseMicrotypeSet[protrusion]{basicmath} % disable protrusion for tt fonts
}{}
\IfFileExists{parskip.sty}{%
\usepackage{parskip}
}{% else
\setlength{\parindent}{0pt}
\setlength{\parskip}{6pt plus 2pt minus 1pt}
}
\usepackage{hyperref}
\hypersetup{
            pdftitle={Laboratory 1},
            pdfauthor={Daniel Alonso},
            pdfborder={0 0 0},
            breaklinks=true}
\urlstyle{same}  % don't use monospace font for urls
\usepackage[margin=1in]{geometry}
\usepackage{color}
\usepackage{fancyvrb}
\newcommand{\VerbBar}{|}
\newcommand{\VERB}{\Verb[commandchars=\\\{\}]}
\DefineVerbatimEnvironment{Highlighting}{Verbatim}{commandchars=\\\{\}}
% Add ',fontsize=\small' for more characters per line
\usepackage{framed}
\definecolor{shadecolor}{RGB}{248,248,248}
\newenvironment{Shaded}{\begin{snugshade}}{\end{snugshade}}
\newcommand{\AlertTok}[1]{\textcolor[rgb]{0.94,0.16,0.16}{#1}}
\newcommand{\AnnotationTok}[1]{\textcolor[rgb]{0.56,0.35,0.01}{\textbf{\textit{#1}}}}
\newcommand{\AttributeTok}[1]{\textcolor[rgb]{0.77,0.63,0.00}{#1}}
\newcommand{\BaseNTok}[1]{\textcolor[rgb]{0.00,0.00,0.81}{#1}}
\newcommand{\BuiltInTok}[1]{#1}
\newcommand{\CharTok}[1]{\textcolor[rgb]{0.31,0.60,0.02}{#1}}
\newcommand{\CommentTok}[1]{\textcolor[rgb]{0.56,0.35,0.01}{\textit{#1}}}
\newcommand{\CommentVarTok}[1]{\textcolor[rgb]{0.56,0.35,0.01}{\textbf{\textit{#1}}}}
\newcommand{\ConstantTok}[1]{\textcolor[rgb]{0.00,0.00,0.00}{#1}}
\newcommand{\ControlFlowTok}[1]{\textcolor[rgb]{0.13,0.29,0.53}{\textbf{#1}}}
\newcommand{\DataTypeTok}[1]{\textcolor[rgb]{0.13,0.29,0.53}{#1}}
\newcommand{\DecValTok}[1]{\textcolor[rgb]{0.00,0.00,0.81}{#1}}
\newcommand{\DocumentationTok}[1]{\textcolor[rgb]{0.56,0.35,0.01}{\textbf{\textit{#1}}}}
\newcommand{\ErrorTok}[1]{\textcolor[rgb]{0.64,0.00,0.00}{\textbf{#1}}}
\newcommand{\ExtensionTok}[1]{#1}
\newcommand{\FloatTok}[1]{\textcolor[rgb]{0.00,0.00,0.81}{#1}}
\newcommand{\FunctionTok}[1]{\textcolor[rgb]{0.00,0.00,0.00}{#1}}
\newcommand{\ImportTok}[1]{#1}
\newcommand{\InformationTok}[1]{\textcolor[rgb]{0.56,0.35,0.01}{\textbf{\textit{#1}}}}
\newcommand{\KeywordTok}[1]{\textcolor[rgb]{0.13,0.29,0.53}{\textbf{#1}}}
\newcommand{\NormalTok}[1]{#1}
\newcommand{\OperatorTok}[1]{\textcolor[rgb]{0.81,0.36,0.00}{\textbf{#1}}}
\newcommand{\OtherTok}[1]{\textcolor[rgb]{0.56,0.35,0.01}{#1}}
\newcommand{\PreprocessorTok}[1]{\textcolor[rgb]{0.56,0.35,0.01}{\textit{#1}}}
\newcommand{\RegionMarkerTok}[1]{#1}
\newcommand{\SpecialCharTok}[1]{\textcolor[rgb]{0.00,0.00,0.00}{#1}}
\newcommand{\SpecialStringTok}[1]{\textcolor[rgb]{0.31,0.60,0.02}{#1}}
\newcommand{\StringTok}[1]{\textcolor[rgb]{0.31,0.60,0.02}{#1}}
\newcommand{\VariableTok}[1]{\textcolor[rgb]{0.00,0.00,0.00}{#1}}
\newcommand{\VerbatimStringTok}[1]{\textcolor[rgb]{0.31,0.60,0.02}{#1}}
\newcommand{\WarningTok}[1]{\textcolor[rgb]{0.56,0.35,0.01}{\textbf{\textit{#1}}}}
\usepackage{graphicx,grffile}
\makeatletter
\def\maxwidth{\ifdim\Gin@nat@width>\linewidth\linewidth\else\Gin@nat@width\fi}
\def\maxheight{\ifdim\Gin@nat@height>\textheight\textheight\else\Gin@nat@height\fi}
\makeatother
% Scale images if necessary, so that they will not overflow the page
% margins by default, and it is still possible to overwrite the defaults
% using explicit options in \includegraphics[width, height, ...]{}
\setkeys{Gin}{width=\maxwidth,height=\maxheight,keepaspectratio}
\setlength{\emergencystretch}{3em}  % prevent overfull lines
\providecommand{\tightlist}{%
  \setlength{\itemsep}{0pt}\setlength{\parskip}{0pt}}
\setcounter{secnumdepth}{0}
% Redefines (sub)paragraphs to behave more like sections
\ifx\paragraph\undefined\else
\let\oldparagraph\paragraph
\renewcommand{\paragraph}[1]{\oldparagraph{#1}\mbox{}}
\fi
\ifx\subparagraph\undefined\else
\let\oldsubparagraph\subparagraph
\renewcommand{\subparagraph}[1]{\oldsubparagraph{#1}\mbox{}}
\fi

% set default figure placement to htbp
\makeatletter
\def\fps@figure{htbp}
\makeatother


\title{Laboratory 1}
\author{Daniel Alonso}
\date{February 7, 2021}

\begin{document}
\maketitle

\begin{Shaded}
\begin{Highlighting}[]
\KeywordTok{library}\NormalTok{(dplyr)}
\end{Highlighting}
\end{Shaded}

\hypertarget{read-data}{%
\section{1. Read data}\label{read-data}}

We first pick a time series from the \emph{series.xls} file attached to
the project and we create a \emph{.csv} file with the selected time
series.

We then select the \emph{bits} column, excluding the time periods.

\begin{Shaded}
\begin{Highlighting}[]
\NormalTok{internet_hits<-}\KeywordTok{read.csv}\NormalTok{(}\DataTypeTok{file=}\StringTok{"./internet_hits.csv"}\NormalTok{, }\DataTypeTok{header=}\OtherTok{TRUE}\NormalTok{)}
\end{Highlighting}
\end{Shaded}

Looking at our data we notice that June 7th is incomplete, as it starts
at 7:00, while the rest of the days start at 00:00, therefore we drop
this day for consistency.

\begin{Shaded}
\begin{Highlighting}[]
\NormalTok{time <-}\StringTok{ }\NormalTok{internet_hits}\OperatorTok{$}\NormalTok{time[}\DecValTok{18}\OperatorTok{:}\KeywordTok{length}\NormalTok{(internet_hits}\OperatorTok{$}\NormalTok{time)]}
\NormalTok{internet_hits <-}\StringTok{ }\NormalTok{internet_hits}\OperatorTok{$}\NormalTok{bits[}\DecValTok{18}\OperatorTok{:}\KeywordTok{length}\NormalTok{(internet_hits}\OperatorTok{$}\NormalTok{bits)]}
\end{Highlighting}
\end{Shaded}

\hypertarget{define-the-time-series-object}{%
\section{2. Define the time series
object}\label{define-the-time-series-object}}

We use the \emph{ts} function to define the time series object. We start
June 8th, 2005 and our frequency is hourly.

\begin{Shaded}
\begin{Highlighting}[]
\NormalTok{hits<-}\KeywordTok{ts}\NormalTok{(internet_hits, }\DataTypeTok{freq=}\DecValTok{24}\NormalTok{)}
\end{Highlighting}
\end{Shaded}

\newpage

\hypertarget{plot-the-time-series}{%
\section{3. Plot the time series}\label{plot-the-time-series}}

We proceed to plot the hourly internet traffic between June 7th and July
28th, 2005.

\hypertarget{first-lets-plot-traffic-for-monday-june-8th-2005}{%
\subsection{First let's plot traffic for Monday June 8th,
2005:}\label{first-lets-plot-traffic-for-monday-june-8th-2005}}

\begin{Shaded}
\begin{Highlighting}[]
\KeywordTok{par}\NormalTok{(}\DataTypeTok{mfrow=}\KeywordTok{c}\NormalTok{(}\DecValTok{1}\NormalTok{,}\DecValTok{1}\NormalTok{))}
\NormalTok{june8th <-}\StringTok{ }\NormalTok{hits[}\DecValTok{1}\OperatorTok{:}\DecValTok{24}\NormalTok{]}
\KeywordTok{plot}\NormalTok{(june8th,}\DataTypeTok{main=}\StringTok{"Hourly internet traffic for June 8th, 2005 in bits"}\NormalTok{)}
\end{Highlighting}
\end{Shaded}

\includegraphics{./figures/unnamed-chunk-5-1.pdf}

We can see that daily traffic peak hours is around lunchtime, between
11:00 and 14:00, showing significantly more traffic than any other time
during the day.

We see that mornings are more active than late night and traffic remains
somewhat consistent between 8:00 and 16:00 with a sharp drop around
17:00 and onwards.

The traffic for this day bottoms at around 4:00, as most people are
probably sleeping.

\newpage

\hypertarget{second-lets-plot-traffic-for-saturday-june-11th-2005}{%
\subsection{Second: let's plot traffic for Saturday June 11th,
2005:}\label{second-lets-plot-traffic-for-saturday-june-11th-2005}}

\begin{Shaded}
\begin{Highlighting}[]
\KeywordTok{par}\NormalTok{(}\DataTypeTok{mfrow=}\KeywordTok{c}\NormalTok{(}\DecValTok{1}\NormalTok{,}\DecValTok{1}\NormalTok{))}
\NormalTok{june11th <-}\StringTok{ }\NormalTok{hits[}\DecValTok{73}\OperatorTok{:}\DecValTok{96}\NormalTok{]}
\KeywordTok{plot}\NormalTok{(june11th,}\DataTypeTok{main=}\StringTok{"Hourly internet traffic for June 11th, 2005 in bits"}\NormalTok{)}
\end{Highlighting}
\end{Shaded}

\includegraphics{./figures/unnamed-chunk-6-1.pdf}

During weekends activity is reduced significantly. We see that the
traffic peak hours now extend up to 19:00, so around 2 hours longer than
weekdays, however, overall, total hourly traffic is around half that of
weekdays with the exception of nights, where hourly traffic is more
similar to that of weekdays.

\newpage

\hypertarget{overall-traffic}{%
\subsection{Overall traffic}\label{overall-traffic}}

\begin{Shaded}
\begin{Highlighting}[]
\KeywordTok{par}\NormalTok{(}\DataTypeTok{mfrow=}\KeywordTok{c}\NormalTok{(}\DecValTok{1}\NormalTok{,}\DecValTok{1}\NormalTok{))}
\KeywordTok{plot}\NormalTok{(hits,}\DataTypeTok{main=}\StringTok{"Hourly internet traffic between June 7th and July 28th, 2005 in bits"}\NormalTok{)}
\end{Highlighting}
\end{Shaded}

\includegraphics{./figures/unnamed-chunk-7-1.pdf}

We can clearly see that the seasonal pattern is repeated weekly, where
weekdays are high traffic and weekends are low traffic days. As soon as
the weekend is over.

\newpage

\hypertarget{logarithmic-transformation}{%
\section{4. logarithmic
transformation}\label{logarithmic-transformation}}

From the plots we can tell the data is homocedastic, therefore we do not
need to perform a log transformation. There's clearly a strong
seasonality aspect to the data which might create a perception of
heterocedasticity in daily timeframes, but not in hourly timeframes.

\hypertarget{first-difference}{%
\section{5. First difference}\label{first-difference}}

The time series does not have a trend, therefore we do not have to take
the first difference.

\hypertarget{seasonal-difference}{%
\section{6. Seasonal difference}\label{seasonal-difference}}

As our data is hourly, we take the 120th-difference (5 days) to attempt
to eliminate seasonality.

\begin{Shaded}
\begin{Highlighting}[]
\NormalTok{ddlhits<-}\KeywordTok{diff}\NormalTok{(dlhits, }\DataTypeTok{lag=}\DecValTok{120}\NormalTok{)}
\KeywordTok{plot}\NormalTok{(ddlhits, }\DataTypeTok{main=}\StringTok{"seasonal series"}\NormalTok{)}
\end{Highlighting}
\end{Shaded}

\includegraphics{./figures/unnamed-chunk-8-1.pdf}

\newpage

\hypertarget{plots}{%
\section{7. plots}\label{plots}}

We plot the seasonal and original series and we obtain the following:

\begin{itemize}
\item
  We can see the data is homocedastic and has no increasing/decreasing
  trend, therefore it is not stationary.
\item
  We also notice that the seasonal series plot does not show any
  pattern, neither trend or seasonality, which we can pretty much deem
  white noise. We can see this time series is stationary.
\end{itemize}

\begin{Shaded}
\begin{Highlighting}[]
\KeywordTok{par}\NormalTok{(}\DataTypeTok{mfrow=}\KeywordTok{c}\NormalTok{(}\DecValTok{2}\NormalTok{,}\DecValTok{1}\NormalTok{))}
\KeywordTok{plot}\NormalTok{(hits, }\DataTypeTok{main=}\StringTok{"original series"}\NormalTok{)}
\KeywordTok{plot}\NormalTok{(ddlhits, }\DataTypeTok{main=}\StringTok{"seasonal series"}\NormalTok{)}
\end{Highlighting}
\end{Shaded}

\includegraphics{./figures/unnamed-chunk-9-1.pdf}

\newpage

\hypertarget{acf-and-pacf}{%
\section{8. acf and pacf}\label{acf-and-pacf}}

Here we plot autocorrelations and partial autocorrelations. This way we
can confirm the accuracy of the patterns we observe in previous steps,
but might be hidden from us.

\includegraphics{./figures/unnamed-chunk-10-1.pdf}

For the acf plot for \emph{hits} we notice that there's a clear pattern,
even clearer than before, we confirm that there is certainly a strong
seasonality to the data. The event repeats itself consistently. The
event has long run memory (in its own timeframe). And we clearly see
that essentially all displayed coefficients are significant, as they all
cross our significance boundaries.

For the pacf plot for \emph{hits} we see that there's only significant
wicks along seasonal limits along with the 2 first wicks.

For the acf and pacf plots for \emph{ddlhits} where we attempt to remove
seasonality we can see significant wicks, but this pattern is much less
clear than in previous plots. Therefore, it's stationary.

\hypertarget{ljung-box-test}{%
\section{9. Ljung-Box test}\label{ljung-box-test}}

For the Ljung-Box test we test the following:

\(H_0\): the model has no dependencies

\(H_1\): the model has dependencies

\begin{Shaded}
\begin{Highlighting}[]
\KeywordTok{Box.test}\NormalTok{(hits,}\DataTypeTok{lag=}\DecValTok{24}\NormalTok{)}
\CommentTok{#> }
\CommentTok{#>  Box-Pierce test}
\CommentTok{#> }
\CommentTok{#> data:  hits}
\CommentTok{#> X-squared = 7129.5, df = 24, p-value < 2.2e-16}
\end{Highlighting}
\end{Shaded}

As to the results looking at our P-vals, for the first test we already
have observed and pointed out that the seasonality is present quite
strongly in the time series. The test has an extremely large test
statistic and a p-value of essentially zero. This, once again, confirms
seasonality.

\begin{Shaded}
\begin{Highlighting}[]
\KeywordTok{Box.test}\NormalTok{(ddlhits,}\DataTypeTok{lag=}\DecValTok{24}\NormalTok{)}
\CommentTok{#> }
\CommentTok{#>  Box-Pierce test}
\CommentTok{#> }
\CommentTok{#> data:  ddlhits}
\CommentTok{#> X-squared = 2037.2, df = 24, p-value < 2.2e-16}
\end{Highlighting}
\end{Shaded}

Our last test shows a significantly decreased test statistic versus the
test conducted previously, however, we still reject the null hypothesis.

\end{document}

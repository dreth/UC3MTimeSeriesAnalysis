\PassOptionsToPackage{unicode=true}{hyperref} % options for packages loaded elsewhere
\PassOptionsToPackage{hyphens}{url}
%
\documentclass[]{article}
\usepackage{lmodern}
\usepackage{amssymb,amsmath}
\usepackage{ifxetex,ifluatex}
\usepackage{fixltx2e} % provides \textsubscript
\ifnum 0\ifxetex 1\fi\ifluatex 1\fi=0 % if pdftex
  \usepackage[T1]{fontenc}
  \usepackage[utf8]{inputenc}
  \usepackage{textcomp} % provides euro and other symbols
\else % if luatex or xelatex
  \usepackage{unicode-math}
  \defaultfontfeatures{Ligatures=TeX,Scale=MatchLowercase}
\fi
% use upquote if available, for straight quotes in verbatim environments
\IfFileExists{upquote.sty}{\usepackage{upquote}}{}
% use microtype if available
\IfFileExists{microtype.sty}{%
\usepackage[]{microtype}
\UseMicrotypeSet[protrusion]{basicmath} % disable protrusion for tt fonts
}{}
\IfFileExists{parskip.sty}{%
\usepackage{parskip}
}{% else
\setlength{\parindent}{0pt}
\setlength{\parskip}{6pt plus 2pt minus 1pt}
}
\usepackage{hyperref}
\hypersetup{
            pdftitle={Laboratory 2},
            pdfauthor={Daniel Alonso},
            pdfborder={0 0 0},
            breaklinks=true}
\urlstyle{same}  % don't use monospace font for urls
\usepackage[margin=1in]{geometry}
\usepackage{color}
\usepackage{fancyvrb}
\newcommand{\VerbBar}{|}
\newcommand{\VERB}{\Verb[commandchars=\\\{\}]}
\DefineVerbatimEnvironment{Highlighting}{Verbatim}{commandchars=\\\{\}}
% Add ',fontsize=\small' for more characters per line
\usepackage{framed}
\definecolor{shadecolor}{RGB}{248,248,248}
\newenvironment{Shaded}{\begin{snugshade}}{\end{snugshade}}
\newcommand{\AlertTok}[1]{\textcolor[rgb]{0.94,0.16,0.16}{#1}}
\newcommand{\AnnotationTok}[1]{\textcolor[rgb]{0.56,0.35,0.01}{\textbf{\textit{#1}}}}
\newcommand{\AttributeTok}[1]{\textcolor[rgb]{0.77,0.63,0.00}{#1}}
\newcommand{\BaseNTok}[1]{\textcolor[rgb]{0.00,0.00,0.81}{#1}}
\newcommand{\BuiltInTok}[1]{#1}
\newcommand{\CharTok}[1]{\textcolor[rgb]{0.31,0.60,0.02}{#1}}
\newcommand{\CommentTok}[1]{\textcolor[rgb]{0.56,0.35,0.01}{\textit{#1}}}
\newcommand{\CommentVarTok}[1]{\textcolor[rgb]{0.56,0.35,0.01}{\textbf{\textit{#1}}}}
\newcommand{\ConstantTok}[1]{\textcolor[rgb]{0.00,0.00,0.00}{#1}}
\newcommand{\ControlFlowTok}[1]{\textcolor[rgb]{0.13,0.29,0.53}{\textbf{#1}}}
\newcommand{\DataTypeTok}[1]{\textcolor[rgb]{0.13,0.29,0.53}{#1}}
\newcommand{\DecValTok}[1]{\textcolor[rgb]{0.00,0.00,0.81}{#1}}
\newcommand{\DocumentationTok}[1]{\textcolor[rgb]{0.56,0.35,0.01}{\textbf{\textit{#1}}}}
\newcommand{\ErrorTok}[1]{\textcolor[rgb]{0.64,0.00,0.00}{\textbf{#1}}}
\newcommand{\ExtensionTok}[1]{#1}
\newcommand{\FloatTok}[1]{\textcolor[rgb]{0.00,0.00,0.81}{#1}}
\newcommand{\FunctionTok}[1]{\textcolor[rgb]{0.00,0.00,0.00}{#1}}
\newcommand{\ImportTok}[1]{#1}
\newcommand{\InformationTok}[1]{\textcolor[rgb]{0.56,0.35,0.01}{\textbf{\textit{#1}}}}
\newcommand{\KeywordTok}[1]{\textcolor[rgb]{0.13,0.29,0.53}{\textbf{#1}}}
\newcommand{\NormalTok}[1]{#1}
\newcommand{\OperatorTok}[1]{\textcolor[rgb]{0.81,0.36,0.00}{\textbf{#1}}}
\newcommand{\OtherTok}[1]{\textcolor[rgb]{0.56,0.35,0.01}{#1}}
\newcommand{\PreprocessorTok}[1]{\textcolor[rgb]{0.56,0.35,0.01}{\textit{#1}}}
\newcommand{\RegionMarkerTok}[1]{#1}
\newcommand{\SpecialCharTok}[1]{\textcolor[rgb]{0.00,0.00,0.00}{#1}}
\newcommand{\SpecialStringTok}[1]{\textcolor[rgb]{0.31,0.60,0.02}{#1}}
\newcommand{\StringTok}[1]{\textcolor[rgb]{0.31,0.60,0.02}{#1}}
\newcommand{\VariableTok}[1]{\textcolor[rgb]{0.00,0.00,0.00}{#1}}
\newcommand{\VerbatimStringTok}[1]{\textcolor[rgb]{0.31,0.60,0.02}{#1}}
\newcommand{\WarningTok}[1]{\textcolor[rgb]{0.56,0.35,0.01}{\textbf{\textit{#1}}}}
\usepackage{graphicx,grffile}
\makeatletter
\def\maxwidth{\ifdim\Gin@nat@width>\linewidth\linewidth\else\Gin@nat@width\fi}
\def\maxheight{\ifdim\Gin@nat@height>\textheight\textheight\else\Gin@nat@height\fi}
\makeatother
% Scale images if necessary, so that they will not overflow the page
% margins by default, and it is still possible to overwrite the defaults
% using explicit options in \includegraphics[width, height, ...]{}
\setkeys{Gin}{width=\maxwidth,height=\maxheight,keepaspectratio}
\setlength{\emergencystretch}{3em}  % prevent overfull lines
\providecommand{\tightlist}{%
  \setlength{\itemsep}{0pt}\setlength{\parskip}{0pt}}
\setcounter{secnumdepth}{0}
% Redefines (sub)paragraphs to behave more like sections
\ifx\paragraph\undefined\else
\let\oldparagraph\paragraph
\renewcommand{\paragraph}[1]{\oldparagraph{#1}\mbox{}}
\fi
\ifx\subparagraph\undefined\else
\let\oldsubparagraph\subparagraph
\renewcommand{\subparagraph}[1]{\oldsubparagraph{#1}\mbox{}}
\fi

% set default figure placement to htbp
\makeatletter
\def\fps@figure{htbp}
\makeatother


\title{Laboratory 2}
\author{Daniel Alonso}
\date{February 16, 2021}

\begin{document}
\maketitle

\hypertarget{use-of-command-arima.sim-to-simulate-arima-processes}{%
\section{1. Use of command ``arima.sim'' to simulate ARIMA
processes}\label{use-of-command-arima.sim-to-simulate-arima-processes}}

\hypertarget{ar1}{%
\subsection{\texorpdfstring{\(AR(1)\)}{AR(1)}}\label{ar1}}

We first simulate an \(AR(1)\) process:

\includegraphics{./figures/unnamed-chunk-1-1.pdf}

\hypertarget{ar2}{%
\subsection{\texorpdfstring{\(AR(2)\)}{AR(2)}}\label{ar2}}

We then simulate an \(AR(2)\) process:

\includegraphics{./figures/unnamed-chunk-2-1.pdf}

\hypertarget{ar4}{%
\subsection{\texorpdfstring{\(AR(4)\)}{AR(4)}}\label{ar4}}

We then simulate an \(AR(4)\) process:

\includegraphics{./figures/unnamed-chunk-3-1.pdf}

\hypertarget{simulates-n-arp-processes-with-n-observationsparameter-phi-standard-deviation-of-a-s}{%
\section{\texorpdfstring{2.simulates N \(AR(p)\) processes \#\#\# With
\(n\) observations,parameter \(\phi\) standard deviation of
\(a = s\)}{2.simulates N AR(p) processes \#\#\# With n observations,parameter \textbackslash{}phi standard deviation of a = s}}\label{simulates-n-arp-processes-with-n-observationsparameter-phi-standard-deviation-of-a-s}}

\tiny

\begin{Shaded}
\begin{Highlighting}[]
\NormalTok{arfun<-}\ControlFlowTok{function}\NormalTok{(N,n,phi,s,c)\{}
\NormalTok{    M=}\KeywordTok{matrix}\NormalTok{(}\DataTypeTok{ncol=}\NormalTok{N,}\DataTypeTok{nrow=}\NormalTok{n)}
    \ControlFlowTok{for}\NormalTok{ (i }\ControlFlowTok{in} \DecValTok{1}\OperatorTok{:}\NormalTok{N)\{}
\NormalTok{        x=}\KeywordTok{arima.sim}\NormalTok{(}\KeywordTok{list}\NormalTok{(}\DataTypeTok{ar=}\NormalTok{phi),}\DataTypeTok{sd=}\NormalTok{s,n)}
\NormalTok{        M[,i]=x}\OperatorTok{+}\NormalTok{c}
\NormalTok{    \}}
    \CommentTok{#Computes mean, variance ACF and PACF}
\NormalTok{    variance=}\KeywordTok{matrix}\NormalTok{(}\DataTypeTok{ncol=}\NormalTok{N,}\DataTypeTok{nrow=}\DecValTok{1}\NormalTok{)}
\NormalTok{    m=}\KeywordTok{matrix}\NormalTok{(}\DataTypeTok{ncol=}\NormalTok{N,}\DataTypeTok{nrow=}\DecValTok{1}\NormalTok{)}
\NormalTok{    rho=}\KeywordTok{matrix}\NormalTok{(}\DataTypeTok{ncol=}\NormalTok{N,}\DataTypeTok{nrow=}\DecValTok{25}\NormalTok{)}
\NormalTok{    pi=}\KeywordTok{matrix}\NormalTok{(}\DataTypeTok{ncol=}\NormalTok{N,}\DataTypeTok{nrow=}\DecValTok{24}\NormalTok{)}
    \ControlFlowTok{for}\NormalTok{(i }\ControlFlowTok{in} \DecValTok{1}\OperatorTok{:}\NormalTok{N)\{}
\NormalTok{        variance[i]=}\KeywordTok{var}\NormalTok{(M[,i])}
\NormalTok{        m[i]=}\KeywordTok{mean}\NormalTok{(M[,i])}
\NormalTok{        r=}\KeywordTok{acf}\NormalTok{(M[,i], }\DataTypeTok{lag.max=}\DecValTok{24}\NormalTok{,}\DataTypeTok{plot=}\OtherTok{FALSE}\NormalTok{)}
\NormalTok{        rho[,i]=r}\OperatorTok{$}\NormalTok{acf}
\NormalTok{        pr=}\KeywordTok{acf}\NormalTok{(M[,i],}\DataTypeTok{lag.max=}\DecValTok{24}\NormalTok{,}\DataTypeTok{type=}\StringTok{"partial"}\NormalTok{,}\DataTypeTok{plot=}\OtherTok{FALSE}\NormalTok{)}
\NormalTok{        pi[,i]=pr}\OperatorTok{$}\NormalTok{acf}
\NormalTok{    \}}
    \CommentTok{#boxplots for the ACF and PACF of lags 1 to 4}
    \KeywordTok{par}\NormalTok{(}\DataTypeTok{mfrow=}\KeywordTok{c}\NormalTok{(}\DecValTok{2}\NormalTok{,}\DecValTok{2}\NormalTok{))}
    \KeywordTok{boxplot}\NormalTok{(rho[}\DecValTok{2}\NormalTok{,],rho[}\DecValTok{3}\NormalTok{,],rho[}\DecValTok{4}\NormalTok{,],rho[}\DecValTok{5}\NormalTok{,], }\DataTypeTok{main=}\StringTok{"ACF coefficients for lags 1 to 4"}\NormalTok{)}
    \KeywordTok{boxplot}\NormalTok{(pi[}\DecValTok{1}\NormalTok{,],pi[}\DecValTok{2}\NormalTok{,],pi[}\DecValTok{3}\NormalTok{,],pi[}\DecValTok{4}\NormalTok{,], }\DataTypeTok{main=}\StringTok{"PACF coefficients for lags 1 to 4"}\NormalTok{)}
    \KeywordTok{plot}\NormalTok{(variance[}\DecValTok{1}\OperatorTok{:}\NormalTok{N],}\DataTypeTok{type=}\StringTok{"l"}\NormalTok{, }\DataTypeTok{main=}\StringTok{"Variance of the generated processes"}\NormalTok{)}
    \KeywordTok{plot}\NormalTok{(m[}\DecValTok{1}\OperatorTok{:}\NormalTok{N],}\DataTypeTok{type=}\StringTok{"l"}\NormalTok{, }\DataTypeTok{main=}\StringTok{"Mean of the generated processes"}\NormalTok{)}
\NormalTok{\}}
\end{Highlighting}
\end{Shaded}

\normalsize

\hypertarget{simulating-ar1-process-with-arfun}{%
\section{\texorpdfstring{Simulating \(AR(1)\) process with
\textbf{arfun}}{Simulating AR(1) process with arfun}}\label{simulating-ar1-process-with-arfun}}

For all simulations we use 2000 iterations of 70 observations and
\(c = 0\).

\hypertarget{using-a-positive-coefficient}{%
\subsection{Using a positive
coefficient}\label{using-a-positive-coefficient}}

\hypertarget{with-phi-0.6.}{%
\subsubsection{\texorpdfstring{With
\(\phi = 0.6\).}{With \textbackslash{}phi = 0.6.}}\label{with-phi-0.6.}}

\begin{Shaded}
\begin{Highlighting}[]
\KeywordTok{arfun}\NormalTok{(}\DecValTok{2000}\NormalTok{, }\DecValTok{70}\NormalTok{, }\FloatTok{0.6}\NormalTok{, }\DecValTok{1}\NormalTok{, }\DecValTok{0}\NormalTok{)}
\end{Highlighting}
\end{Shaded}

\includegraphics{./figures/unnamed-chunk-5-1.pdf}

We can see that the first few ACF coefficients are still showing a
ladder-like structure. The median of the first boxplot in both the ACF
and PACF plots is of almost \textasciitilde{}0.6.

The variance is not huge, and from testing we know that it's less and
less significantly variable as the number of samples and simulations
increase simultaneously. Bias is also reduced significantly as these
values increase. In fact, for a very large number of simulations, we
notice no significant ACF or PACF coefs.

The means oscillate around -0.3 and 0.3, while the variance goes to just
under 1 up to a bit over 3.

\hypertarget{using-a-negative-coefficient}{%
\subsection{Using a negative
coefficient}\label{using-a-negative-coefficient}}

\hypertarget{with-phi--0.6}{%
\subsubsection{\texorpdfstring{With
\(\phi = -0.6\)}{With \textbackslash{}phi = -0.6}}\label{with-phi--0.6}}

\begin{Shaded}
\begin{Highlighting}[]
\KeywordTok{arfun}\NormalTok{(}\DecValTok{2000}\NormalTok{, }\DecValTok{70}\NormalTok{, }\FloatTok{-0.6}\NormalTok{, }\DecValTok{1}\NormalTok{, }\DecValTok{0}\NormalTok{)}
\end{Highlighting}
\end{Shaded}

\includegraphics{./figures/unnamed-chunk-6-1.pdf}

The pattern for the ACF plot oscillates as the powers of a negative
value change sign depending on whether the exponent is even or odd. For
the PACF coefficients, only the first one is significant. For the first
PACF coefficient with a negative \emph{ar}, the value is pretty much
\textasciitilde{} -0.6.

The means oscillate around -0.2 and 0.2 with some outliers, while the
variance is between 0.5 and 3.5 with most of the cases between 1 and 3.

Here, in contrast to a positive \(\phi\), the significance of our ACF
coefficients is not exactly smoothed down, and ACF coefficients remain
significant, while only the first PACF coefficient is significant.

\newpage

\hypertarget{simulating-ar3-process-with-arfun}{%
\section{\texorpdfstring{Simulating \(AR(3)\) process with
\textbf{arfun}}{Simulating AR(3) process with arfun}}\label{simulating-ar3-process-with-arfun}}

I have decided to include an \(AR(3)\) process to comparatively see what
happens to significance in ACF and PACF plots, along with variance and
mean, when the signs of \(\phi\) change, with multiple \(\phi\). All
simulations are performed using 2000 iterations, 70 observations and
\(c = 0\).

\hypertarget{using-3-positive-coefficients}{%
\subsection{Using 3 positive
coefficients}\label{using-3-positive-coefficients}}

\hypertarget{with-phi_1-0.3-phi_2-0.5-and-phi_3-0.1}{%
\subsubsection{\texorpdfstring{With \(\phi_1 = 0.3, \phi_2 = 0.5\) and
\(\phi_3 = 0.1\)}{With \textbackslash{}phi\_1 = 0.3, \textbackslash{}phi\_2 = 0.5 and \textbackslash{}phi\_3 = 0.1}}\label{with-phi_1-0.3-phi_2-0.5-and-phi_3-0.1}}

\begin{Shaded}
\begin{Highlighting}[]
\KeywordTok{arfun}\NormalTok{(}\DecValTok{2000}\NormalTok{, }\DecValTok{70}\NormalTok{, }\KeywordTok{c}\NormalTok{(}\FloatTok{0.3}\NormalTok{, }\FloatTok{0.5}\NormalTok{, }\FloatTok{0.1}\NormalTok{), }\DecValTok{1}\NormalTok{, }\DecValTok{0}\NormalTok{)}
\end{Highlighting}
\end{Shaded}

\includegraphics{./figures/unnamed-chunk-7-1.pdf}

In this case we see ACF significance drops slowly vs the \(AR(1)\) case.
For PACF only maybe our first and second coefficients are significant.

The variance has much more extreme values and hovers (mostly) between 1
and 4, however, there's plenty of peaks, and all much more significant
than those of the \(AR(1)\) case.

For the mean we notice a similar trend, however values tend to reach
outside the interval \(y = (1,-1)\) much more often

For a significantly higher amount of iterations, we notice that we
increase the chances for high variance events to occur.

\hypertarget{using-2-positive-and-1-negative-coefficients}{%
\subsection{Using 2 positive and 1 negative
coefficients}\label{using-2-positive-and-1-negative-coefficients}}

\hypertarget{with-phi_1--0.3-phi_2-0.5-and-phi_3-0.1}{%
\subsubsection{\texorpdfstring{With \(\phi_1 = -0.3, \phi_2 = 0.5\) and
\(\phi_3 = 0.1\)}{With \textbackslash{}phi\_1 = -0.3, \textbackslash{}phi\_2 = 0.5 and \textbackslash{}phi\_3 = 0.1}}\label{with-phi_1--0.3-phi_2-0.5-and-phi_3-0.1}}

\begin{Shaded}
\begin{Highlighting}[]
\KeywordTok{arfun}\NormalTok{(}\DecValTok{2000}\NormalTok{, }\DecValTok{70}\NormalTok{, }\KeywordTok{c}\NormalTok{(}\OperatorTok{-}\FloatTok{0.3}\NormalTok{, }\FloatTok{0.5}\NormalTok{, }\FloatTok{0.2}\NormalTok{), }\DecValTok{1}\NormalTok{, }\DecValTok{0}\NormalTok{)}
\end{Highlighting}
\end{Shaded}

\includegraphics{./figures/unnamed-chunk-8-1.pdf}

In the case where \(\phi_1\) is negative while the rest are postive, we
notice the variance is significantly decreased, while our ACF and PACF
coefficients remain significant for the first two lags. However they
drop down quite quickly for the subsequent lags.

\hypertarget{using-1-positive-and-2-negative-coefficients}{%
\subsection{Using 1 positive and 2 negative
coefficients}\label{using-1-positive-and-2-negative-coefficients}}

\hypertarget{with-phi_1--0.3-phi_2--0.5-and-phi_3-0.1}{%
\subsubsection{\texorpdfstring{With \(\phi_1 = -0.3, \phi_2 = -0.5\) and
\(\phi_3 = 0.1\)}{With \textbackslash{}phi\_1 = -0.3, \textbackslash{}phi\_2 = -0.5 and \textbackslash{}phi\_3 = 0.1}}\label{with-phi_1--0.3-phi_2--0.5-and-phi_3-0.1}}

\begin{Shaded}
\begin{Highlighting}[]
\KeywordTok{arfun}\NormalTok{(}\DecValTok{2000}\NormalTok{, }\DecValTok{70}\NormalTok{, }\KeywordTok{c}\NormalTok{(}\OperatorTok{-}\FloatTok{0.3}\NormalTok{, }\FloatTok{-0.5}\NormalTok{, }\FloatTok{0.2}\NormalTok{), }\DecValTok{1}\NormalTok{, }\DecValTok{0}\NormalTok{)}
\end{Highlighting}
\end{Shaded}

\includegraphics{./figures/unnamed-chunk-9-1.pdf}

This option also keeps variance lower, overall, while maybe slightly
higher than the previous option. The mean is significantly less variable
overall.

The ACF and PACF coefficients seem to be boosted for the second lag vs
the previous allocation of \(\phi\). Significance of such coefficients
oscillates, as usual when using any negative \(\phi\), however it drops
quite quickly for PACF, where the second lag is more significant than
the first one.

\hypertarget{using-3-negative-coefficients}{%
\subsection{Using 3 negative
coefficients}\label{using-3-negative-coefficients}}

\hypertarget{with-phi_1--0.3-phi_2--0.5-and-phi_3--0.1}{%
\subsubsection{\texorpdfstring{With \(\phi_1 = -0.3, \phi_2 = -0.5\) and
\(\phi_3 = -0.1\)}{With \textbackslash{}phi\_1 = -0.3, \textbackslash{}phi\_2 = -0.5 and \textbackslash{}phi\_3 = -0.1}}\label{with-phi_1--0.3-phi_2--0.5-and-phi_3--0.1}}

\begin{Shaded}
\begin{Highlighting}[]
\KeywordTok{arfun}\NormalTok{(}\DecValTok{2000}\NormalTok{, }\DecValTok{70}\NormalTok{, }\KeywordTok{c}\NormalTok{(}\OperatorTok{-}\FloatTok{0.3}\NormalTok{, }\FloatTok{-0.5}\NormalTok{, }\FloatTok{-0.2}\NormalTok{), }\DecValTok{1}\NormalTok{, }\DecValTok{0}\NormalTok{)}
\end{Highlighting}
\end{Shaded}

\includegraphics{./figures/unnamed-chunk-10-1.pdf}

When using 3 negative coefficients we yield an even lower mean, and much
less variant one as well. Our variance is overall also significantly
reduced. Significance for ACF coefficients drops in general as well,
making them all perhaps less significant. Similarly with PACF
coefficients, with the exception of perhaps the 2nd lag.

\hypertarget{playing-with-values-slightly}{%
\subsection{Playing with values
slightly}\label{playing-with-values-slightly}}

\hypertarget{with-phi_1--0.3-phi_2--0.1-and-phi_3--0.5}{%
\subsubsection{\texorpdfstring{With \(\phi_1 = -0.3, \phi_2 = -0.1\) and
\(\phi_3 = -0.5\)}{With \textbackslash{}phi\_1 = -0.3, \textbackslash{}phi\_2 = -0.1 and \textbackslash{}phi\_3 = -0.5}}\label{with-phi_1--0.3-phi_2--0.1-and-phi_3--0.5}}

\begin{Shaded}
\begin{Highlighting}[]
\KeywordTok{arfun}\NormalTok{(}\DecValTok{2000}\NormalTok{, }\DecValTok{70}\NormalTok{, }\KeywordTok{c}\NormalTok{(}\OperatorTok{-}\FloatTok{0.3}\NormalTok{, }\FloatTok{-0.1}\NormalTok{, }\FloatTok{-0.5}\NormalTok{), }\DecValTok{1}\NormalTok{, }\DecValTok{0}\NormalTok{)}
\end{Highlighting}
\end{Shaded}

\includegraphics{./figures/unnamed-chunk-11-1.pdf}

Changing which \(\phi\) is larger.smaller also yields an interesting
result, where the 1st and 3rd lags seem to show the most significance
for both ACF and PACF plots.

\hypertarget{with-phi_1--0.3-phi_2--0.1-and-phi_3--0.5-1}{%
\subsubsection{\texorpdfstring{With \(\phi_1 = -0.3, \phi_2 = -0.1\) and
\(\phi_3 = -0.5\)}{With \textbackslash{}phi\_1 = -0.3, \textbackslash{}phi\_2 = -0.1 and \textbackslash{}phi\_3 = -0.5}}\label{with-phi_1--0.3-phi_2--0.1-and-phi_3--0.5-1}}

\begin{Shaded}
\begin{Highlighting}[]
\KeywordTok{arfun}\NormalTok{(}\DecValTok{2000}\NormalTok{, }\DecValTok{70}\NormalTok{, }\KeywordTok{c}\NormalTok{(}\OperatorTok{-}\FloatTok{0.1}\NormalTok{, }\FloatTok{-0.1}\NormalTok{, }\FloatTok{-0.5}\NormalTok{), }\DecValTok{1}\NormalTok{, }\DecValTok{0}\NormalTok{)}
\end{Highlighting}
\end{Shaded}

\includegraphics{./figures/unnamed-chunk-12-1.pdf}

Leveling down \(\phi_1\) and \(\phi_2\) to be the same value and also
much higher than \(\phi_3\) basically removes the significance of the
first lag.

\end{document}
